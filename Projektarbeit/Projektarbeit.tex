\documentclass[a4paper,singleside,12pt,titlepage]{scrartcl}

\usepackage[left=2.5cm, right=2.5cm]{geometry}
\usepackage[utf8]{inputenc}
\usepackage[T1]{fontenc}
\usepackage[ngerman]{babel}
\usepackage[onehalfspacing]{setspace}
\usepackage{amsmath}
\usepackage{amssymb}
\usepackage{csquotes}
\usepackage{hyperref}
\usepackage{algorithm}
\usepackage{algpseudocode}
\usepackage{graphicx}
\usepackage{wrapfig}
\usepackage{tikz}
\usepackage{circuitikz}
\usepackage{tikz-timing}
\usepackage{bytefield}
\usepackage{pgf-umlcd}
\usepackage[backend=biber, style=ieee, citestyle=ieee]{biblatex}
\usepackage{rotating}

\bibliography{Projektarbeit.bib}

\newcommand{\linkstyle}[1]{\color{blue}#1}
\newcommand{\link}[2]{\href{#1}{\linkstyle{#2}}}
\newcommand{\secref}[2]{\hyperref[sec:#1]{\linkstyle{\texttt{#2}}}}

\begin{document}
	    \begin{titlepage}
		
		\hspace*{0pt}
		\vfill
		\begin{center}
			{\Huge\textbf{Implementation eines Touchscreen GUI auf einem STM32F769I-Discovery Board}}\par
			\bigskip
			\bigskip
			{\large Hochschule Bonn-Rhein-Sieg, Fachbereich Informatik}\par
			{\large Projekt-Seminar Mikrocontroller (WiSe 2021/2022)}\par
			\bigskip
			\bigskip
			\bigskip
			\bigskip
			\bigskip
			\textsc{Neo Hornberger, Tim Ludwig}\par
		\end{center}
		\vfill
		\hspace*{0pt}
	\end{titlepage}

	\tableofcontents
	\listoffigures
	\newpage
	
	\section{Einleitung}
		Wir haben uns in diesem Projekt mit typischen Komponenten eines \emph{Touchscreen User Interfaces}
		und der Umsetzung eines solchen Systems auf einem Mikrocontroller-Modul (dem STM32F769I-Disco Board) beschäftigt.
		Im Folgenden werden die relevanten Komponenten des Mikrocontroller-Moduls beschrieben
		und die Umsetzung eines \emph{Proof-of-Concept} Touch GUI dargestellt.
		
		\subsection{Aufgabenteilung}
			Da wir nur ein Discovery-Board zur Verfügung hatten, haben wir das Projekt ursprünglich wie folgt aufgeteilt:\\
			Funktionsweise relevanter Komponenten und Protokolle: \textsc{Neo Hornberger}\\
			Aufbau und Implementation der Software: \textsc{Tim Ludwig}\\
			Wie von einer Teamarbeit zu erwarten, waren diese Grenzen aber eher fließend.
			Insbesondere das Verständnis einiger Protokolle ist auch für die Entwicklung der Software wichtig,
			während eine \emph{zweite Meinung}\footnote{siehe auch \link{https://rubberduckdebugging.com/}{Rubber Duck Debugging}}
			natürlich auch hilfreich war.
	
	\section{}
	\input{sections/lcd.tex}
	\subsection{I²C}

% I²C Abkürzung erklären

I²C basiert auf einem Bussystem.
Dies bedeutet, dass alle Komponenten, die miteinander kommunizieren sollen, an einer oder mehreren gemeinsamen Leitungen anliegen.
Der I²C-Bus besteht aus zwei Leitungen.
Die Datenleitung wird \abbrevtt{SDA}{serial data line} und die Taktleitung wird \abbrevtt{SCL}{serial clock line} genannt.

Alle Daten, die versendet werden, sind in Bytes aufgeteilt.
Jedes Byte wird in \abbrev{MSB}{most significant bit}[\itshape]-Reihenfolge auf die Datenleitung geschrieben.

Um eine bestimmte Komponente über den Bus ansprechen zu können, benötigt jede Komponente eine eindeutige Adresse, welche von dem Hersteller zugewiesen wird.

Zudem wird zur Kommunikation das \texttt{Controller}/\texttt{Target}-Prinzip verwendet.
Die sogenannten \texttt{Controller} sind Komponenten, die eine Kommunikation mit einer anderen Komponente initiieren können.
Sogenannte \texttt{Targets} sind Komponenten, die von einem \texttt{Controller} angesprochen werden können.
Der \texttt{Controller} einer Kommunikation entscheidet außerdem, ob er einem \texttt{Target} Daten senden oder bestimmte Daten von diesem anfordern möchte.
Während der Kommunikation erzeugt immer der \texttt{Controller} den Takt, welcher über die Taktleitung verteilt wird.

Trotz der Möglichkeit mehrere \texttt{Controller} an den Bus anzuschließen und kommunizieren zu lassen, wird im Weiteren davon ausgegangen, dass nur ein \texttt{Controller} an dem Bus angeschlossen ist.

\begin{wrapfigure}{r}{.4\textwidth}
	\centering
	\scalebox{1.25}{\begin{figure}
	\begin{tikztimingtable}
		% width=11.375
		%
		SDA & H L E 6{D{}} E L .375H ; [dotted] h \\
		SCL & 1.5H .875C {c N(c1) c} {c N(c2) c} {c N(c3) c} {c N(c4) c} {c N(c5) c} {c N(c6) c} {c N(c7) c} {c N(c8) c} {c N(c9) c} ; [dotted] l \\
		%
		\begin{extracode}[every node/.style={font=\tiny}]
			\foreach \i in {1,...,9}
				\node[left=-.345em of c\i.mid](\i){\i};
		\end{extracode}
	\end{tikztimingtable}
\end{figure}
}
	
	\caption{Kommunikationsaufbau}
	\figcite{I2C-spec_userManual}
	\label{i2c-start}
\end{wrapfigure}
Um eine Kommunikation zu starten, sendet der \texttt{Controller} eine Startanweisung, die 7-Bit Adresse des anzusprechenden \texttt{Targets} und ein Bit, welches beschreibt, ob der \texttt{Controller} Daten schreiben oder anfordern möchte.
Eine Startanweisung ist definiert als eine fallende Flanke auf der Datenleitung, während die Taktleitung auf \texttt{HIGH} steht.
Wenn das \texttt{Target} seine Adresse wiedererkannt hat, bestätigt es den Aufbau der Kommunikation durch ein sogenanntes \texttt{Acknowledge}.
Um ein solches Signal erzeugen zu können, gibt der Sender die Datenleitung frei und der Empfänger setzt diese auf \texttt{LOW}.

\begin{wrapfigure}{r}{.4\textwidth}
	\centering
	\scalebox{1.25}{\begin{tikztimingtable}
	% width=9.625
	%
	SDA & ; [dotted] h ; .25H E 6{D{}} E N(ackL) L N(ackR) .375H ; [dotted] h \\
	SCL & ; [dotted] l ; .625L {c N(c1) c} {c N(c2) c} {c N(c3) c} {c N(c4) c} {c N(c5) c} {c N(c6) c} {c N(c7) c} {c N(c8) c} {c N(c9) c} ; [dotted] l \\
	%
	\begin{extracode}[every node/.style={font=\tiny}]
		\foreach \i in {1,...,9}
			\node[left=-.345em of c\i.mid] (\i) {\i};
		
		\begin{pgfonlayer}{background}
			\draw[draw=gray,dashed] ([shift={(-.075,-2.5)}] ackL.LOW) rectangle ([shift={(.125,.5)}] ackR.high);
			\node[color=gray,below left=.5ex and .25em of c9.low,anchor=north] {\texttt{ACK}};
		\end{pgfonlayer}
	\end{extracode}
\end{tikztimingtable}
}
	
	\caption{Datenübertragung}
	\figcite{I2C-spec_userManual}
	\label{i2c-data}
\end{wrapfigure}
Wenn der \texttt{Controller} Daten anfordern möchte, wird dieser nach dem \texttt{Acknowledge} zum Empfänger und das \texttt{Target} zum Sender.
Sobald der Kommunikationsaufbau bestätigt wurde, fängt der Sender mit dem nächsten Taktzyklus an die einzelnen Bytes über den Bus zu versenden.
Hierbei wird vom Empfänger nach jedem Byte eine Empfangsbestätigung erzeugt.

\begin{wrapfigure}{r}{.4\textwidth}
	\centering
	\scalebox{1.25}{\begin{figure}
	\begin{tikztimingtable}
		% width=2
		%
		SDA & ; [dotted] .25E .75L ; L H \\
		SCL & ; [dotted] .75E .25C ; 2H \\
	\end{tikztimingtable}
\end{figure}
}
	\scalebox{1.25}{\begin{figure}
	\begin{tikztimingtable}
		SDA & ; [dotted] .5H ; .75H L \\
		SCL & ; [dotted] .5L ; .25L 1.5H \\
	\end{tikztimingtable}
\end{figure}
}
	
	\caption{Kommunikationsenden}
	\figcite{I2C-spec_userManual}
	\label{i2c-stop}
\end{wrapfigure}
Um die Kommunikation zu beenden, sendet der \texttt{Controller} entweder eine Anweisung zur Beendigung der Kommunikation oder eine erneute Startanweisung.
Eine Anweisung zur Beendigung der Kommunikation ist definiert als eine steigende Flanke auf der Datenleitung, während die Taktleitung auf \texttt{HIGH} steht.
Eine erneute Startanweisung fasst das Ende der laufenden und den Start einer neuen Kommunikation zusammen.
Diese verkürzte Methode wird bevorzugt, wenn der \texttt{Controller} aus dem Schreibmodus in den Lesemodus oder umgekehrt wechseln möchte.

\cite{I2C-spec_userManual}

	\subsection{Touchscreen}

Der verwendete Touchscreen ist ein projiziert-kapazitiver Touchscreen, welcher mit einer Eigenkapazität arbeitet. \cite[\itshape4.4~Touch~panel]{ts-userManual}

Neben den projiziert-kapazitiven Touchscreens existieren auch noch Oberflächen-kapazitive Touchscreens, welche leicht anders funktionieren.
%Bei beiden Varianten wird eine Berührung durch Änderung von Kapazitäten im Touchscreen erkannt.

Bei Oberflächen-kapazitiven Touchscreens wird eine Glasscheibe mit einer sehr dünnen Metallschicht beschichtet.
An allen vier Ecken wird eine kleine Wechselspannung angelegt, welche ein schwaches elektrisches Feld erzeugt.
Wenn nun ein leitfähiges Material die Oberfläche berührt, wird dies ge- und entladen.
Durch das Messen des Stroms durch die Ecken kann nun herausgefunden werden, wie weit die Berührung von einer bestimmten Ecke entfernt ist.
Durch Kombination der vier Entfernungen zu den vier Ecken kann die Position der Berührung ermittelt werden.
\cite[\itshape8~Findings~and~Discussion]{ts-holzinger}

Bei projiziert-kapazitiven Touchscreens hingegen wird ein Raster aus leitfähigen Streifen verwendet, welches unter der Glasscheibe liegt.
Hierbei dient eine Ebene des Rasters als Sensorebene und die andere Ebene als Maskenebene.
Die Streifen der Maskenebene können entweder mit \texttt{ground} oder mit der \textit{active backplane} verbunden werden.
Wenn einer dieser Streifen mit \texttt{ground} verbunden wurde, können Streifen der Sensorebene auf der Höhe des Maskenstreifens keine Berührung erkennen.
Um alle Berührungspositionen zu ermitteln, werden nun die Maskenstreifen linear sequentiell mit der \textit{active backplane} verbunden, sodass alle außer einem Streifen mit \texttt{ground} verbunden sind.
Durch dieses Verfahren ist für jedes Abfrageintervall eine Achse fest und die andere durch den Nutzer beeinflussbar.
Die veränderte Kapazität an der Position der Berührung gibt Aufschluss darüber, wie weit entfernt diese Berührung von einem Sensorstreifen liegt.
Somit können auch Berührungen zwischen zwei Sensorstreifen erkannt werden.
\cite{ts-self_capacitive_multitouch}

Sogenannte \textit{"ghost touches"} können entstehen, wenn statt der beschriebenen Vorgehensweise alle Maskenstreifen mit der \textit{active backplane} verbunden sind und somit die gesamte Länge der Sensorstreifen Berührungen erkennen kann.
Bei einer Abfrage mit zwei Berührungspunkten ergeben sich zwei mögliche Ermittlungsergebnisse.
Das eine Ergebnis besteht aus den eigentlichen Berührungspositionen $(x_0,y_0)$ und $(x_1,y_1)$.
Das andere besteht aus den falsch ermittelten Positionen $(x_0,y_1)$ und $(x_1,y_0)$.

Der verwendete Touchscreen kann mit dem Mikrocontroller über einen I²C-Bus kommunizieren und somit alle notwendigen Daten an diesen übermitteln.
Zudem können verschiedene Einstellungen gemacht werden.
Diese Daten finden sich an spezifizierten Adressen des Daughterboards.
Eine kleine tabellarische Darstellung der im Code abgefragten oder gesetzten Adressen und deren Bedeutung:
\medskip
\begin{figure}[h!]
	\centering
	\begin{figure}
	\newcommand{\bitformat}{\color{gray}\small}
	\newcommand{\boxformat}[1]{%
		\centering\small%
		\let\boxwidth=\width%
		\raisebox{0pt}[\heightof{X}][0pt]{#1}%
	}
	\newcommand{\bitboxempty}[2][lr]{\bitbox[#1]{#2}{}}
	\newcommand{\wordboxempty}[2][lr]{\wordbox[#1]{#2}{}}
	\newenvironment{addressrow}[1]{%
		\begin{leftwordgroup}{%
				\bitformat%
				\makebox[\widthof{0xmm}][l]{#1}%
			}%
		}{%
		\end{leftwordgroup}%
	}
	\newcommand{\resizetext}[1]{\resizebox{\boxwidth}{!}{~#1~}}
	
	\begin{bytefield}[bitwidth=.115\textwidth,
		bitformatting=\bitformat,
		boxformatting=\boxformat,
		leftcurly=.,
		leftcurlyspace=0pt]{8}
		\bitheader{0-7}\\
		\begin{addressrow}{0x00}
			\bitboxempty[ltr]{2} & \bitbox{1}{STATUS} & \bitbox{2}{P1\_X} & \bitbox{2}{P1\_Y} & \bitboxempty[ltr]{1}
		\end{addressrow}\\
		\begin{addressrow}{0x08}
			\bitboxempty{1} & \bitbox{2}{P2\_X} & \bitbox{2}{P2\_Y} & \bitboxempty{3}
		\end{addressrow}\\
		\begin{addressrow}{\hspace{\widthof{0x}}\vdots}
			\wordboxempty{2}
		\end{addressrow}\\
		\begin{addressrow}{0xA0}
			\bitboxempty[lrb]{4} & \bitbox{1}{\resizetext{IT\_MODE}} & \bitboxempty[lrb]{3}
		\end{addressrow}\\
	\end{bytefield}
\end{figure}

	%
	\caption{Touchscreen Adresstabelle (verwendete Adressen)}
	\figcite{stm32-ts}
	\label{ts-address_table}
\end{figure}

	\subsection{Aufbau des Touchscreen UI's}
	Nun also zur Software. Kern des Systems ist eine Abstraktion des Touchscreens auf dem Daughterboard:
	Die Klasse \secref{touchscreen_class}{TouchScreen}.
	
	
	\subsubsection{Die Klasse TouchScreen}\label{sec:touchscreen_class}
		Diese Klasse bietet verschiedene Methoden, welche die Kommunikation mit dem Daughterboard vereinfachen.
		Um das System nicht unnötig aus zu lasten ist sie in der Lage, einen GPIO-Interrupt zu behandeln.
		Eine Instanz dieser Klasse speichert die aktuellen Touch-Koordinaten und löst ein \secref{EventSystem}{Event} aus, wenn eine Berührung erkannt wird.

		\medskip
		\texttt{refresh()} fragt über den I²C BUS die aktuellen Touch-Koordinaten ab und speichert diese.
		Da das Koordinatensystem des TouchScreens nicht mit dem des LCD übereinstimmt, werden die über I²C gelesenen Koordinaten
		in das Koordinatensystem des LCD transformiert.

		\medskip
		\texttt{interruptHandler()} behandelt einen GPIO-Interrupt auf dem eingestellten Port.
		Diese Methode nutzt \texttt{refresh()} um die Koordinaten zu aktualisieren. Sie speichert die alten Werte lokal zwischen,
		um das richtige \secref{EventSystem}{Event} zu erzeugen. Dann wird die \texttt{onEvent(\&TouchEvent)} Methode
		des konfigurierten \secref{components}{Wurzelcontainers} aufgerufen.
		
	
	\subsubsection{Komponenten und Container}\label{sec:components}
	
	\subsubsection{Event System}\label{sec:EventSystem}
	
	\cite{stm32-base}
	\cite{stm32-ts}
	\cite{ts-holzinger}
	
	\newpage
	\printbibliography[heading=bibnumbered]
\end{document}
