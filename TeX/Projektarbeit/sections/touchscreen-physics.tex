\subsection{Touchscreen}

Der verwendete Touchscreen ist ein projiziert-kapazitiver Touchscreen, welcher mit einer Eigenkapazität arbeitet. \cite{ts-userManual}

Neben den projiziert-kapazitiven Touchscreens existieren auch noch Oberflächen-kapazitive Touchscreens, welche leicht anders funktionieren.
%Bei beiden Varianten wird eine Berührung durch Änderung von Kapazitäten im Touchscreen erkannt.

Bei Oberflächen-kapazitiven Touchscreens wird eine Glasscheibe mit einer sehr dünnen Metallschicht beschichtet.
An allen vier Ecken wird eine kleine Wechselspannung angelegt, welche ein schwaches elektrisches Feld erzeugt.
Wenn nun ein leitfähiges Material die Oberfläche berührt, wird dies ge- und entladen.
Durch das Messen des Stroms durch die Ecken kann nun herausgefunden werden, wie weit die Berührung von einer bestimmten Ecke entfernt ist.
Durch Kombination der vier Entfernungen zu den vier Ecken kann die Position der Berührung ermittelt werden.
\cite{ts-holzinger}

Bei projiziert-kapazitiven Touchscreens hingegen wird ein Raster aus leitfähigen Streifen verwendet, welches unter der Glasscheibe liegt.
Hierbei dient eine Ebene des Rasters als Sensorebene und die andere Ebene als Maskenebene.
Die Streifen der Maskenebene können entweder mit \texttt{ground} oder mit der \textit{active backplane} verbunden werden.
Wenn einer dieser Streifen mit \texttt{ground} verbunden wurde, können Streifen der Sensorebene auf jener Höhe keinen Berührung erkennen.
\cite{ts-self_capacitive_multitouch}
