\subsection{Touchscreen}

Der verwendete Touchscreen ist ein projiziert-kapazitiver Touchscreen, welcher mit einer Eigenkapazität arbeitet. \cite{ts-userManual}

Neben den projiziert-kapazitiven Touchscreens existieren auch noch Oberflächen-kapazitive Touchscreens, welche leicht anders funktionieren.
%Bei beiden Varianten wird eine Berührung durch Änderung von Kapazitäten im Touchscreen erkannt.

Bei Oberflächen-kapazitiven Touchscreens wird eine Glasscheibe mit einer sehr dünnen Metallschicht beschichtet.
An allen vier Ecken wird eine kleine Wechselspannung angelegt, welche ein schwaches elektrisches Feld erzeugt.
Wenn nun ein leitfähiges Material die Oberfläche berührt, wird an dieser Stelle das elektrische Feld unterbrochen.
Durch das Messen des Stroms durch die Ecken kann nun herausgefunden werden, wie weit die Berührung von einer bestimmten Ecke entfernt ist.
Durch Kombination der vier Entfernungen zu den vier Ecken kann die Position der Berührung ermittelt werden.
\cite{ts-holzinger}

Bei projiziert-kapazitiven Touchscreens hingegen wird aus leitfähigen Platten ein Raster gebaut, welches unter der Glasscheibe liegt.
\cite{ts-self_capacitive_multitouch}
