\subsection{I²C}

% I²C Abkürzung erklären

I²C basiert auf einem Bussystem.
Dies bedeutet, dass alle Komponenten, die miteinander kommunizieren sollen, an einer oder mehreren gemeinsamen Leitungen anliegen.
Der I²C-Bus besteht aus zwei Leitungen.
Die Datenleitung wird \abbrevtt{SDA}{serial data line} und die Taktleitung wird \abbrevtt{SCL}{serial clock line} genannt.

Alle Daten, die versendet werden, sind in 8-Bit-Wörter (Bytes) aufgeteilt.
Jedes Bit wird nacheinander in \abbrev{MSB}{most significant bit}[\itshape]-Bitreihenfolge auf die Datenleitung geschrieben.

Um eine bestimmte Komponente über den Bus ansprechen zu können, benötigt jede Komponente eine eindeutige Adresse, welche von den Herstellern zugewiesen wird.

Zudem wird zur Kommunikation das \texttt{Controller/Target}-Prinzip verwendet.
Die sogenannten \texttt{Controller} sind Komponenten, die eine Kommunikation mit einer anderen Komponente initiieren können.
Sogenannte \texttt{Targets} sind Komponenten, die von einem \texttt{Controller} angesprochen werden können.
Der \texttt{Controller} einer Kommunikation entscheidet außerdem, ob er einem \texttt{Target} Daten senden oder bestimmte Daten von diesem anfordern möchte.
Während der Kommunikation generiert jener den Takt, welcher über die Taktleitung verteilt wird.

Im Weiteren wird davon ausgegangen, dass nur ein Controller am Bus angeschlossen ist.

Um eine Kommunikation zu starten, sendet der \texttt{Controller} eine Startanweisung, die 7-Bit Adresse des \texttt{Targets} und ein Bit, welches beschreibt, ob der \texttt{Controller} Daten schreiben oder anfordern möchte.
Eine Startanweisung ist definiert als eine senkende Flanke auf der Datenleitung, während die Taktleitung auf \texttt{HIGH} steht.
Wenn der \texttt{Target} seine Adresse wiedererkannt hat, bestätigt er den Aufbau der Kommunikation durch ein sogenanntes \texttt{Acknowledge}.
Damit der Empfang eines Bytes bestätigen kann, gibt der Sender die Datenleitung frei und der Empfänger setzt diese auf \texttt{LOW}.
