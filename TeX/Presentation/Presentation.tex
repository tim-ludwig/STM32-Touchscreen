\documentclass[12pt, aspectratio=169]{beamer}

\usepackage[utf8]{inputenc}
\usepackage[T1]{fontenc}
\usepackage[ngerman]{babel}
\usepackage{amsmath}
\usepackage{amssymb}
\usepackage{csquotes}
\usepackage{hyperref}
\usepackage{algorithm}
\usepackage{algpseudocode}
\usepackage{graphicx}
\usepackage{wrapfig}
\usepackage{tikz}
\usepackage{circuitikz}
\usepackage{tikz-timing}
\usepackage{bytefield}
\usepackage{pgf-umlcd}
\usepackage[backend=biber, style=ieee, citestyle=ieee]{biblatex}
\usepackage{rotating}
\usepackage{colortbl}
\usepackage{xargs}
\usepackage{caption}
\usepackage[export]{adjustbox}

\usetikzlibrary{positioning,trees}

\usetikztiminglibrary{advnodes,either,beamer,overlays}

\newcommand{\linkstyle}[1]{\color{blue}#1}
\newcommand{\link}[2]{\href{#1}{\linkstyle{#2}}}
\newcommand{\secref}[2]{\hyperref[sec:#1]{\linkstyle{\texttt{#2}}}}

\newcommand{\figcite}[2][\itshape\footnotesize]{\vspace*{-2ex}\caption*{#1 nach \cite{#2}}\vspace*{-2ex}}

\newcommand{\setupbytefield}{%
	\newcommand{\bitformat}{\color{gray}\small}
	\newcommand{\boxformat}[1]{%
		\centering\small%
		\let\boxwidth=\width%
		\raisebox{0pt}[\heightof{X}][0pt]{##1}%
	}
	\newcommand{\bitboxempty}[2][lr]{\bitbox[##1]{##2}{}}
	\newcommand{\wordboxempty}[2][lr]{\wordbox[##1]{##2}{}}
	\newenvironment{addressrow}[1]{%
		\begin{leftwordgroup}{%
				\bitformat%
				\makebox[\widthof{0xmm}][l]{##1}%
			}%
		}{%
		\end{leftwordgroup}%
	}
	\newcommand{\resizetext}[1]{\resizebox{\boxwidth}{!}{~##1~}}
}

\newcommand{\textoverline}[1]{$\overline{\mbox{#1}}$}

\newcommandx{\abbrev}[5][1=,3=\bfseries,5=]{{#1#2}\footnote{{#3#2} {#5#4}}}
\newcommandx{\abbrevtt}[4][1=\bfseries,4=\itshape]{\abbrev[\ttfamily]{#2}[#1]{#3}[#4]}


\usetheme{Madrid}
\usecolortheme{whale}

\newcounter{listnum}

\title{Mikrocontroller Touchscreen GUI}
\author{Neo Hornberger, Tim Ludwig}
\date{21.01.2022}

\bibliography{../Bibliography.bib}

\begin{document}
	\maketitle
	
	\section{Touchscreen Hardware}
	\frame{\tableofcontents[currentsection]}
	
	\begin{frame}{Resistive Touchscreens}
		\begin{columns}
			\begin{column}{0.4\textwidth}
				\begin{figure}
					\includegraphics[max width=.9\textwidth,max height=.9\textheight]{../Images/ResistiveTouchScreen.png}
					%
					\caption*{\tiny von \link{https://commons.wikimedia.org/wiki/File:Touchscreen.png}{Wikimedia Commons}}
				\end{figure}
			\end{column}
			\begin{column}{0.6\textwidth}
				Funktionsweise: 
				\begin{itemize}
					\item<+-> Kontakt zweier Schichten durch Deformierung
					\item<+-> Anlegen einer Spannung über eine Schicht
					\item<+-> Messen des Potentials an den Enden der anderen Schicht
					\item<+-> Wdh. mit vertauschten Rollen
				\end{itemize}
				
				Häufige Nutzung:
				\begin{itemize}
					\item Kiosksysteme
					\item Industrie
					\item \dots
				\end{itemize}
			\end{column}
		\end{columns}
	\end{frame}

	\begin{frame}{kapazitive Touchscreens}
		Arten von kapazitiven Touchscreens:
		\begin{itemize}
			\item Oberflächen-kapazitiv
			\item Projiziert-kapazitiv
		\end{itemize}
		\medskip
		Häufige Nutzung:
		\begin{itemize}
			\item Smartphones
			\item Tablets
			\item \dots
		\end{itemize}
	\end{frame}

	\begin{frame}{Oberflächen-kapazitive Touchscreens}
		\begin{columns}
			\begin{column}{0.4\textwidth}
				\begin{figure}
					\includegraphics[max width=.9\textwidth,max height=.9\textheight]{../Images/SurfaceCapacitiveTouchScreen.png}
					%
					\caption*{\tiny von \link{https://commons.wikimedia.org/wiki/File:TouchScreen_capacitive.svg}{Wikimedia Commons}}
				\end{figure}
			\end{column}
			\begin{column}{0.6\textwidth}
				Funktionsweise:
				\begin{itemize}
					\item<+-> Wechselspannung an den Ecken anlegen
					\item<+-> bei Berührung wird eine Kapazität auf- und entladen
					\item<+-> den Strom durch die Ecken messen
				\end{itemize}
			\end{column}
		\end{columns}
	\end{frame}
	
	\begin{frame}{Projiziert-kapazitive Touchscreens}
		\begin{columns}
			\begin{column}{0.4\textwidth}
				\begin{figure}
					\includegraphics[max width=.9\textwidth,max height=.7\textheight]{../Images/ProjectedCapacitiveTouchScreen.png}
					%
					\caption*{\tiny von \link{https://commons.wikimedia.org/wiki/File:TouchScreen_projective_capacitive.svg}{Wikimedia Commons}}
				\end{figure}
			\end{column}
			\begin{column}{0.6\textwidth}
				Funktionsweise:
				\begin{itemize}
					\item<+-> zwei Ebenen Gitter aus Leitern
					\item<+-> Anlegen einer Wechselspannung an eine Ebene
					\item<+-> Messen des durch die Leiter der anderen Ebene fließenden Stroms
				\end{itemize}
			\end{column}
		\end{columns}
	\end{frame}
	
	\begin{frame}{Kapazitiv vs. Resistiv}
		\centering
		\begin{tabular}{| >{\columncolor{structure.fg!80}\color{white}} l | p{.4\textwidth} | p{.4\textwidth} |}
			\hline
			\rowcolor{structure.fg!80} & \color{white}Vorteile & \color{white}Nachteile\\\hline
			kapazitiv & \normalcolor
				\begin{itemize}
					\item geringer Verschleiß
					\item Multitouch
				\end{itemize} &
				\begin{itemize}
					\item leitendes Material (Finger, spez. Stifte, \dots) notwendig
					\item Kalibration
				\end{itemize}\\\hline
			resistiv &
				\begin{itemize}
					\item ohne leitendes Material bedienbar
					\item unempfindlich gegenüber Staub, Wasser, \dots
				\end{itemize} &
				\begin{itemize}
					\item erhöhter Verschleiß
					\item unerwünschtes Auslösen
				\end{itemize}\\\hline
		\end{tabular}
	\end{frame}

	\section{I²C}
	\frame{\tableofcontents[currentsection]}
	
	\begin{frame}{Der I²C-Bus}
		\begin{itemize}
			\item Bussystem bestehend aus zwei Leitungen (\texttt{SDA}, \texttt{SCL})
			\item Aufteilung der Daten in Bytes
			\item Versenden der Bytes in MSB-Reihenfolge
			\item Komponenten besitzen eine 7-Bit Adresse
			\item Controller/Target-Prinzip
			\item nach jedem gesendeten Byte bestätigt der Empfänger dies
		\end{itemize}
	\end{frame}
	
	\begin{frame}{Kommunikationsstart}
		\begin{columns}
			\begin{column}{0.45\textwidth}
				\texttt{Controller}:
				\begin{enumerate}
					\item Startanweisung (\texttt{START})
					\item 7-Bit Adresse
					\item R/\textoverline{W}-Bit
					
					\setcounter{listnum}{\value{enumi}}
				\end{enumerate}
				\texttt{Target}:
				\begin{enumerate}
					\setcounter{enumi}{\value{listnum}}
					
					\item Bestätigung (\texttt{ACK})
				\end{enumerate}
			\end{column}
			\begin{column}{0.5\textwidth}
				\begin{figure}
					\begin{figure}
	\begin{tikztimingtable}
		SDA & H L 7{D{}} D L .25H ; [dotted] .5H \\
		SCL & 1.75H ; [!wscale=.5!] 9{L H} ; .5L ; [dotted] .5L \\
	\end{tikztimingtable}
\end{figure}

				\end{figure}
			\end{column}
		\end{columns}
	\end{frame}

	\begin{frame}{Übertragung von Daten}
		\begin{columns}
			\begin{column}{0.45\textwidth}
				Sender:
				\begin{enumerate}
					\item 8-Bit Wort
					
					\setcounter{listnum}{\value{enumi}}
				\end{enumerate}
				
				Empfänger:
				\begin{enumerate}
					\setcounter{enumi}{\value{listnum}}
					
					\item Bestätigung (\texttt{ACK})
				\end{enumerate}
				
				\vspace{5ex}
				Wdh. bis alle Bytes versendet sind.
			\end{column}
			\begin{column}{0.5\textwidth}
				\begin{figure}
					\begin{figure}
	\begin{tikztimingtable}
		SDA & ; [dotted] .5H ; .25H 7{D{}} D L .25H ; [dotted] .5H \\
		SCL & ; [dotted] .5L ; [!wscale=.5!] 9{L H} ; .5L ; [dotted] .5L \\
	\end{tikztimingtable}
\end{figure}

				\end{figure}
			\end{column}
		\end{columns}
	\end{frame}

	\begin{frame}{Kommunikationsende}
		\begin{columns}
			\begin{column}{0.55\textwidth}
				\texttt{Controller}:
				\begin{enumerate}[a]
					\item Stoppanweisung (\texttt{STOP})
					\item erneute Startanweisung (\texttt{repeated START})
				\end{enumerate}
			\end{column}
			\begin{column}{0.4\textwidth}
				\begin{figure}
					\begin{figure}
	\begin{tikztimingtable}
		SDA & ; [dotted] .5L ; .75L H \\
		SCL & ; [dotted] .5L ; .25L 1.5H \\
	\end{tikztimingtable}
\end{figure}

				\end{figure}
				\begin{figure}
					\begin{figure}
	\begin{tikztimingtable}
		% width=2
		%
		SDA & ; [dotted] H ; H L \\
		SCL & ; [dotted] e c ; 2H \\
	\end{tikztimingtable}
\end{figure}

				\end{figure}
			\end{column}
		\end{columns}
	\end{frame}
	
	\section{Software}
	\frame{\tableofcontents[currentsection]}
	
	\begin{frame}{Planung \& Entwurfsziele}
		Implementation typischer Touchscreen GUI Komponenten in C++.
		\pause
		\bigskip
		\begin{itemize}
			\item leicht benutzbar $\Rightarrow$ orientiert an typischen Design Guidelines \cite{material-components} \pause
			\item leicht programmierbar $\Rightarrow$ viel Abstraktion, wenig Boilerplate \pause
			\item möglichst Ressourcen schonend $\Rightarrow$ Abfragen und zeichnen nur wenn notwendig
		\end{itemize}
	\end{frame}
	
	\subsection{Klassen}
	\begin{frame}{BoundingBox}
		\centering
		\begin{tikzpicture}
			\begin{class}[text width=11cm]{BoundingBox}{0, 0}
	\attribute{- left, top, right, bottom: WORD}
	
	\operation{+ BoundingBox(x: WORD, y: WORD, width: WORD, height: WORD)}
	\operation{+ contains(x: WORD, y: WORD): bool}
	\operation{+ resizeToFit(x: WORD, y: WORD): void}
	\operation{+ resizeToFit(box: BoundingBox\&): void}
	\operation{+ x(): WORD}
	\operation{+ y(): WORD}
	\operation{+ width(): WORD}
	\operation{+ height(): WORD}
\end{class}
		\end{tikzpicture}
	\end{frame}
	
	\begin{frame}{Component}
		\centering
		\begin{tikzpicture}
			\begin{abstractclass}[text width=11cm]{Component}{0, 0}
	\attribute{\# box: BoundingBox}
	
	\operation{+ Component(x: WORD, y: WORD, w: WORD, h: WORD)}
	\operation{+ getBoundingBox(): BoundingBox\&}
	\operation[0]{+ onEvent(event: DragEvent\&): void}
	\operation[0]{+ onEvent(event: TouchEvent\&): void}
	\operation[0]{+ onEvent(event: ReleaseEvent\&): void}
	\operation[0]{+ show(display: cDevDispalyGraphic\&): void}
\end{abstractclass}
		\end{tikzpicture}
	\end{frame}

	\begin{frame}{Container}
		\centering
		\begin{tikzpicture}
			\begin{abstractclass}[text width=3cm]{Component}{0, 0}
			\end{abstractclass}
			
			\begin{class}[text width=9cm]{Container}{0, -2.5}
	\inherit{Component}
	
	\attribute{- compontents: list<Component*>}
	
	\operation{+ addComponent(comp: Component*): void}
	\operation{+ removeComponent(comp: Component*): void}
\end{class}
		\end{tikzpicture}
	\end{frame}

	\begin{frame}{Slider}
		\centering
		\begin{tikzpicture}
			\begin{abstractclass}[text width=3cm]{Component}{-4, 0}
			\end{abstractclass}
			
			\node[right=of Component] (image) {\includegraphics[height=2cm]{../Images/Slider.png}};
			
			\begin{class}[text width=11cm]{Slider}{0, -2.5}
	\inherit{Component}
	
	\attribute{- position: double}
	\attribute{- listeners: list<function<void(SliderEvent\&)>\relax>}
	
	\operation{- setPosition(x: WORD): void}
	\operation{+ getPosition(): double}
	\operation{+ addEventListener(listener: function<void(SliderEvent\&)>): void}
\end{class}
		\end{tikzpicture}
	\end{frame}

	\begin{frame}{Toggle Switch}
		\centering
		\begin{tikzpicture}
			\begin{abstractclass}[text width=3cm]{Component}{-4, 0}
			\end{abstractclass}
			
			\node[right=of Component] (image) {\includegraphics[height=2cm]{../Images/Switch.png}};
			
			\begin{class}[text width=11cm]{ToggleSwitch}{0, -2.5}
	\inherit{Component}
	
	\attribute{- state: bool}
	\attribute{- listeners: list<function<void(ToggleSwitchEvent\&)>\relax>}
	
	\operation{+ getState(): bool}
	\operation{+ addEventListener(listener: function<void(ToggleSwitchEvent\&)>): void}
\end{class}
		\end{tikzpicture}
	\end{frame}

	\begin{frame}{Radio Button}
		\centering
		\begin{tikzpicture}
			\begin{abstractclass}[text width=3cm]{Component}{-4, 0}
			\end{abstractclass}
			
			\node[right=of Component] (image) {\includegraphics[height=2cm]{../Images/RadioButton.png}};
			
			\begin{class}[text width=11cm]{RadioButton}{0, -2.5}
	\inherit{Component}
	
	\attribute{- state: bool}
	\attribute{- deselectable: bool}
	\attribute{- listeners: list<function<void(RadioButtonEvent\&)>\relax>}
	
	\operation{+ getState(): bool}
	\operation{+ addEventListener(listener: function<void(RadioButtonEvent\&)>): void}
\end{class}
		\end{tikzpicture}
	\end{frame}

	\begin{frame}{TouchScreen}
		\centering\scalebox{0.8}{
			\begin{tikzpicture}
				\begin{class}[text width=11cm]{TouchScreen}{0, 0}
	\attribute{- x, y: WORD}
	\attribute{- touchCount: WORD}
	\attribute{- width, height: WORD}
	\attribute{- i2cTouch: cHwI2Cmaster::Device\&}
	\attribute{- interruptPort: cHwPort\_N::PortId}
	\attribute{+ interruptPin: BYTE}
	\attribute{+ rootContainer: Container*}
	\attribute{+ \emph{static} INSTANCE: TouchScreen*}
	
	\operation{+ refresh(): void}
	\operation{+ setInterruptMode(mode: bool): void}
	\operation{+ interruptHandler(): void}
	\operation{+ setRootContainer(c: Container*): void}
	\operation{+ getRootContainer(): Container*}
\end{class}
			\end{tikzpicture}
		}
	\end{frame}

	\begin{frame}{Beispiel GUI}
		\centering\begin{tikzpicture}[sibling distance=7em,
	every node/.style = {shape=rectangle, rounded corners,
		draw, align=center}]
	\node {TouchScreen}
		child { node {RootContainer}
			child { node {Switch}}
			child { node {Switch}}
			child { node {Container}
				child { node {RadioButton}}
				child { node {RadioButton}}
				child { node {RadioButton}}
			}
			child { node {Slider}}
		};
\end{tikzpicture}
	\end{frame}

	\begin{frame}{Das Event System}
		\begin{columns}
			\begin{column}{0.6\textwidth}
				\begin{tikzpicture}[y=-1cm]
	\node[draw](root){RootContainer};
	\node[draw=gray,color=gray,below right=1em of root,anchor=north west](slider){Slider};
	\node[draw=gray,color=gray,below right=1em of slider,anchor=north west](listener1){Listener 1};
	\node[draw,below left=6ex and 0em of slider,anchor=north west](switch){Switch};
	\node[draw,below right=1em of switch,anchor=north west](listener2){Listener 2};
	\node[draw,below left=1ex and 0em of listener2,anchor=north west](listener3){Listener 3};
	
	
	\draw[-latex,color=gray] (root.south) |- (slider);
	\draw[-latex,color=gray] (slider.south) |- (listener1);
	\draw[-latex] (root.south) |- (switch);
	\draw[-latex] (switch.south) |- (listener2);
	\draw[-latex] (switch.south) |- (listener3);
\end{tikzpicture}
			\end{column}
			\begin{column}{0.4\textwidth}
				\pause
				Berührung außerhalb der \emph{Bounding-Box} von \texttt{Slider}.
			\end{column}
		\end{columns}
	\end{frame}
	
	\begin{frame}{Offene ToDo's}
		\begin{itemize}
			\item Zu- und Wegschalten einzelner Komponenten
			\item Performance
			\item weitere Komponenten
			\item \dots
		\end{itemize}
	\end{frame}
	
	\section{Quellen}
	\begin{frame}{Quellen}
		\nocite{ts-holzinger}
		\nocite{I2C-spec_userManual}
		\nocite{stm32_refManual}
		\printbibliography
	\end{frame}
\end{document}
